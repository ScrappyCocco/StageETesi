%----------------------------------------------------------------------------------------
%   USEFUL COMMANDS
%----------------------------------------------------------------------------------------

\newcommand{\dipartimento}{Dipartimento di Matematica ``Tullio Levi-Civita''}

%----------------------------------------------------------------------------------------
% 	USER DATA
%----------------------------------------------------------------------------------------

% Data di approvazione del piano da parte del tutor interno; nel formato GG Mese AAAA
% compilare inserendo al posto di GG 2 cifre per il giorno, e al posto di 
% AAAA 4 cifre per l'anno
\newcommand{\dataApprovazione}{Data}

\input{DatiSensibili.tex}

% Dati del Tutor Interno (Docente)
\newcommand{\titoloTutorInterno}{Prof.}
\newcommand{\nomeTutorInterno}{NomeDocente}
\newcommand{\cognomeTutorInterno}{CognomeDocente}

\newcommand{\prospettoSettimanale}{
     % Personalizzare indicando in lista, i vari task settimana per settimana
     % sostituire a XX il totale ore della settimana
    \begin{itemize}
        \item \textbf{Prima Settimana - 8-12 Giugno 2020 (40 ore)}
        \begin{itemize}
            \item Installazione VTK;
            \item Installazione Slicer3D;
            \item Studio funzionalità librerie;
        \end{itemize}
        \item \textbf{Seconda Settimana - 15-19 Giugno 2020 (32 ore)} 
        \begin{itemize}
            \item Analisi QML per possibile integrazione;
            \item Progettazione e inizio sviluppo GUI base in Qt;
        \end{itemize}
        \item \textbf{Terza Settimana - 22-26 Giugno 2020 (32 ore)} 
        \begin{itemize}
        	\item Fine sviluppo GUI base in Qt;
            \item Progettazione e impostazione Pipeline 3D;
        \end{itemize}
        \item \textbf{Quarta Settimana - 29 Giugno - 3 Luglio 2020 (40 ore)} 
        \begin{itemize}
            \item Integrazione in GUI Qt;
            \item Continuazione sviluppo Pipeline;
        \end{itemize}
        \item \textbf{Quinta Settimana - 6-10 Luglio 2020 (40 ore)} 
        \begin{itemize}
        	\item Analisi ottimizzazione GPU;
            \item Sviluppo editor delle funzioni di trasferimento;
        \end{itemize}
        \item \textbf{Sesta Settimana - 13-17 Luglio 2020 (40 ore)} 
        \begin{itemize}
            \item Sviluppo strumenti di clipping del volume;
        \end{itemize}
        \item \textbf{Settima Settimana - 20-24 Luglio 2020 (40 ore)} 
        \begin{itemize}
            \item Sviluppo prodotto;
        \end{itemize}
        \item \textbf{Ottava Settimana - 27-31 Luglio 2020 (40 ore)} 
        \begin{itemize}
            \item Collaudo prodotto;
            \item Conclusione stesura della documentazione;
        \end{itemize}
    \end{itemize}
}

% Indicare il totale complessivo (deve essere compreso tra le 300 e le 320 ore)
\newcommand{\totaleOre}{304}

\newcommand{\obiettiviObbligatori}{
	 \item \underline{\textit{O01}}: visualizzazione interattiva volumetrica;
	 \item \underline{\textit{O02}}: possibilità di scelta della funzione di trasferimento dei voxel (colore, trasparenza);
	 
}

\newcommand{\obiettiviDesiderabili}{
	 \item \underline{\textit{D01}}: piani di taglio del volume;
	 \item \underline{\textit{D02}}: modifiche alla funzione taglio;
	 \item \underline{\textit{D03}}: ottimizzazione rendering GPU;
}

\newcommand{\obiettiviFacoltativi}{
	 \item \underline{\textit{F01}}: algoritmi di segmentazione con ITK;
	 \item \underline{\textit{F02}}: analisi unit-testing su GUI-Qt;
	 \item \underline{\textit{F03}}: porting librerie aziendali su CMake;
}