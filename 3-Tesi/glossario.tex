%**************************************************************
% Glossario
%**************************************************************
%\renewcommand{\glossaryname}{Glossario}

\newglossaryentry{dicomg}
{
    name=\glslink{dicomg}{DICOM},
    text=DICOM,
    sort=dicom,
    description={Con il termine \textit{DICOM} è uno standard che definisce i criteri per la comunicazione, la visualizzazione, l'archiviazione e la stampa di informazioni di tipo biomedico quali ad esempio immagini radiologiche.}
}

\newglossaryentry{opensg}
{
    name=\glslink{opensg}{Open Source},
    text=open source,
    sort=open source,
    description={Con il termine \textit{open source} si fa riferimento ad un \textit{software} la cui licenza permette di utilizzarlo, modificarlo e redistribuirlo}
}

\newglossaryentry{frameworkg}
{
    name=\glslink{frameworkg}{Framework},
    text=Framework,
    sort=framework,
    description={Con il termine \textit{Framework} si intende un'architettura logica di supporto (spesso un'implementazione logica di un particolare design pattern) sul quale un software può essere progettato e realizzato, spesso facilitandone lo sviluppo da parte del programmatore.}
}

\newglossaryentry{widgetg}
{
    name=\glslink{widgetg}{Widget},
    text=Widget,
    sort=widget,
    description={Con il termine \textit{Widget}, in Qt, si intende l'oggetto base da cui costruire un elemento per l'interfaccia grafica, come un bottone o una casella di testo.}
}

%**************************************************************
% Acronimi
%**************************************************************
\renewcommand{\acronymname}{Acronimi e abbreviazioni}

\newacronym[description={\glslink{dicomg}{Digital Imaging and COmmunications in Medicine}}]
    {dicom}{Digital Imaging and COmmunications in Medicine}{DICOM}