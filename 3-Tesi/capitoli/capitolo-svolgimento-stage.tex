% !TEX encoding = UTF-8
% !TEX TS-program = pdflatex
% !TEX root = ../tesi.tex

%**************************************************************
\chapter{Resoconto Stage}
\label{cap:resoconto-stage}
%**************************************************************

%**************************************************************
\section{Pianificazione}
\subsection{Pianificazione iniziale}

\subsection{Sistema di Issue e Gantt}

\subsection{Discussioni e incontri con il tutor}

\subsection{Problemi e ritardi}

%**************************************************************
\section{Implementazione}
\subsection{Installazione e compilazione delle librerie}
\intro{La preparazione dell'ambiente di sviluppo e le librerie necessarie + Studio Slicer?}\\

\subsection{Primo CMakeLists}
\intro{Il primo file CMakeLists per fare il build con Cmake}\\

\subsection{Integrazione Qt-VTK}
\intro{Le prime prove nell'utilizzare VTK con Qt}\\

\subsection{Primo prototipo}
\intro{Il primo prototipo funzionante di visualizzatore}\\

\subsection{Aggiunta strumenti}
\intro{Gli strumenti aggiunti per interagire con il volume, tra cui Mark}\\

\subsection{Compilazione librerie aziendali}
\intro{Le librerie aziendali da utilizzare e i problemi nel compilarle}\\

\subsection{Problemi librerie aziendali}
\intro{I problemi nell'utilizzare le librerie aziendali con VTK}\\

\subsection{Modifiche interfaccia}
\intro{Com'è stata modificata l'interfaccia per renderla semplice e portatile}\\

%**************************************************************
\section{Test}
\subsection{Possibili test su una UI}
\intro{I problemi nel testare un'interfaccia grafica}\\

\subsection{Test di 3D Slicer e VTK}
\intro{Come 3D Slicer e VTK fanno questi test}\\

\subsection{Test implementati}
\intro{Discussione sui test fatti su librerie statiche e "model"}\\