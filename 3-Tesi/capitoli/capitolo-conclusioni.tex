% !TEX encoding = UTF-8
% !TEX TS-program = pdflatex
% !TEX root = ../tesi.tex

%**************************************************************
\chapter{Conclusioni}
\label{cap:conclusioni}
%**************************************************************

%**************************************************************
\section{Consuntivo finale}
Come previsto nel piano di lavoro, sono state svolte 304 ore in totale. Tuttavia, come accennato nella sezione \nameref{sec:problemi-ritardi} (§\ref{sec:problemi-ritardi}) alcune attività hanno richiesto più tempo di quanto preventivato. Il consuntivo finale è visibile nella tabella \ref{table:consuntivo-finale}.

\begin{center}
    \begin{table}[h]
    \def\arraystretch{1.5}
    \begin{tabular}{| c | c | c | p{5cm} |}
        \hline
        \textbf{Preventivo} & \textbf{Consuntivo} & \textbf{Differenza} & \textbf{Attività} \\ \hline  
        75 & 75 & 0 & Installazione e studio delle librerie necessarie\\ \hline
        26 & 20 & -6 & Studio e progettazione GUI in Qt\\ \hline
        9 & 14 & +5 & Sviluppo GUI in Qt\\ \hline
        45 & 35 & -10 & Analisi e progettazione della Pipeline 3D\\ \hline
        24 & 30 & +6 & Sviluppo Pipeline 3D\\ \hline
        85 & 90 & +5 & Sviluppo del prodotto\\ \hline
        25 & 20 & -5 & Collaudo prodotto\\ \hline
        15 & 20 & +5 & Stesura documentazione finale\\ \hline
    \end{tabular}
    \caption{Consuntivo finale}
    \label{table:consuntivo-finale}
    \end{table}
\end{center}

%**************************************************************
\section{Raggiungimento degli obiettivi}
	 \item \underline{\textit{O01}}: visualizzazione interattiva volumetrica;
	 \item \underline{\textit{O02}}: possibilità di scelta della funzione di trasferimento dei voxel (colore, trasparenza);
	 
}

	 \item \underline{\textit{D01}}: piani di taglio del volume;
	 \item \underline{\textit{D02}}: modifiche alla funzione taglio;
	 \item \underline{\textit{D03}}: ottimizzazione rendering GPU;

	 \item \underline{\textit{F01}}: algoritmi di segmentazione con ITK;
	 \item \underline{\textit{F02}}: analisi unit-testing su GUI-Qt;
	 \item \underline{\textit{F03}}: porting librerie aziendali su CMake;

%**************************************************************
\section{Conoscenzee abilità acquisite}

%**************************************************************
\section{Valutazione personale}
