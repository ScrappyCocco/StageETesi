% !TEX encoding = UTF-8
% !TEX TS-program = pdflatex
% !TEX root = ../tesi.tex

%**************************************************************
\chapter{Conclusioni}
\label{cap:conclusioni}
%**************************************************************

%**************************************************************
\section{Consuntivo finale}
Come previsto nel piano di lavoro, sono state svolte 304 ore in totale. Tuttavia, come accennato nella sezione \nameref{sec:problemi-ritardi} (§\ref{sec:problemi-ritardi}), alcune attività hanno richiesto più tempo di quanto preventivato. Il consuntivo finale è visibile nella tabella \ref{table:consuntivo-finale}.

\begin{center}
    \begin{table}[h]
    \def\arraystretch{1.5}
    \begin{tabular}{| c | c | c | p{5cm} |}
        \hline
        \textbf{Preventivo} & \textbf{Consuntivo} & \textbf{Differenza} & \textbf{Attività} \\ \hline  
        75 & 75 & 0 & Installazione e studio delle librerie necessarie\\ \hline
        26 & 20 & -6 & Studio e progettazione GUI in Qt\\ \hline
        9 & 14 & +5 & Sviluppo GUI in Qt\\ \hline
        45 & 35 & -10 & Analisi e progettazione della Pipeline 3D\\ \hline
        24 & 30 & +6 & Sviluppo Pipeline 3D\\ \hline
        85 & 90 & +5 & Sviluppo del prodotto\\ \hline
        25 & 20 & -5 & Collaudo prodotto\\ \hline
        15 & 20 & +5 & Stesura documentazione finale\\ \hline
    \end{tabular}
    \caption{Consuntivo finale}
    \label{table:consuntivo-finale}
    \end{table}
\end{center}

%**************************************************************
\section{Raggiungimento degli obiettivi}
\subsection{Obiettivi obbligatori}
\begin{itemize}
\item \underline{\textit{O01}}: visualizzazione interattiva volumetrica: soddisfatto. Il visualizzatore volumetrico realizzato funziona come richiesto e permette di caricare, visualizzare ed interagire con un volume;
\item \underline{\textit{O02}}: possibilità di scelta della funzione di trasferimento dei voxel (colore, trasparenza): soddisfatto. C'è una lista di funzioni di trasferimento predefinite tra cui scegliere.
\end{itemize}

\subsection{Obiettivi desiderabili}
\begin{itemize}
\item \underline{\textit{D01}}: piani di taglio del volume: soddisfatto. \'E possibile tagliare il volume tramite piani di taglio con un box o con un piano singolo;
\item \underline{\textit{D02}}: modifiche alla funzione taglio: soddisfatto. I widget per tagliare il volume permettono l'interazione dell'utente per spostarli o modificarli;
\item \underline{\textit{D03}}: ottimizzazione rendering GPU: soddisfatto. Il mapper GPU con alcune impostazioni particolari è il metodo più performante per effettuare il render del volume.
\end{itemize}

\subsection{Obiettivi facoltativi}
\begin{itemize}
	 \item \underline{\textit{F01}}: algoritmi di segmentazione con ITK: non soddisfatto. Non è stato trovato il tempo necessario per studiare approfonditamente ITK, in quanto utilizzare gli algoritmi di segmentazione era solo una possibilità non richiesta dall'azienda;
	 \item \underline{\textit{F02}}: analisi unit-testing su GUI-Qt: soddisfatto, sono stati sviluppati alcuni semplici unit-test utilizzando il framework di test fornito da Qt;
	 \item \underline{\textit{F03}}: porting librerie aziendali su CMake: soddisfatto, le librerie aziendali rilevanti sono state portate su CMake, e molte attivamente utilizzate.
\end{itemize}

%**************************************************************
\section{Conoscenze e abilità acquisite}
\subsection{Immagini radiologiche e DICOM}

\subsection{Volume Rendering}

\subsection{Linguaggi e tecnologie}

\subsection{Strumenti}

%**************************************************************
\section{Valutazione personale}
