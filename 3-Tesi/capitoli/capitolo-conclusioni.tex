% !TEX encoding = UTF-8
% !TEX TS-program = pdflatex
% !TEX root = ../tesi.tex

%**************************************************************
\chapter{Conclusioni}
\label{cap:conclusioni}
%**************************************************************

%**************************************************************
\section{Consuntivo finale}
Come previsto nel piano di lavoro, sono state svolte 304 ore in totale. Tuttavia, come accennato nella sezione \nameref{sec:problemi-ritardi} (§\ref{sec:problemi-ritardi}), alcune attività hanno richiesto più tempo di quanto preventivato. Inoltre, la stesura di tutta la documentazione è stata più impegnativa del previsto, vista la mole di aspetti da documentare, mentre il collaudo ha richiesto una quantità di tempo leggermente minore. Il consuntivo finale è visibile nella tabella \ref{table:consuntivo-finale}.

\begin{center}
    \begin{table}[h]
    \def\arraystretch{1.5}
    \setlength\extrarowheight{5pt}
    \begin{tabular}{| c | c | c | p{5cm} |}
        \hline
        \textbf{Preventivo} & \textbf{Consuntivo} & \textbf{Differenza} & \textbf{Attività} \\ \hline  
        75 & 75 & 0 & Installazione e studio delle librerie necessarie\\ \hline
        26 & 20 & -6 & Studio e progettazione GUI in Qt\\ \hline
        9 & 14 & +5 & Sviluppo GUI in Qt\\ \hline
        45 & 35 & -10 & Analisi e progettazione della \emph{Pipeline} 3D\\ \hline
        24 & 30 & +6 & Sviluppo \emph{Pipeline} 3D\\ \hline
        85 & 90 & +5 & Sviluppo del prodotto\\ \hline
        25 & 20 & -5 & Collaudo prodotto\\ \hline
        15 & 20 & +5 & Stesura documentazione finale\\ \hline
    \end{tabular}
    \caption{Consuntivo finale}
    \label{table:consuntivo-finale}
    \end{table}
\end{center}

%**************************************************************
\section{Raggiungimento degli obiettivi}
Nella sezione \nameref{sec:descrizione-obiettivi} (§\ref{sec:descrizione-obiettivi}) è possibile visualizzare gli obiettivi discussi e definiti prima di iniziare l'attività di stage. Ora che lo stage è concluso possiamo verificarne il grado di completezza.

\subsection{Obiettivi obbligatori}
\begin{itemize}
\item \underline{\textit{O01}} (visualizzazione interattiva volumetrica): soddisfatto. Il visualizzatore volumetrico realizzato funziona come richiesto e permette di caricare, visualizzare ed interagire con un volume;
\item \underline{\textit{O02}} (possibilità di scelta della funzione di trasferimento dei \emph{voxel}): soddisfatto. C'è una lista di funzioni di trasferimento predefinite tra cui scegliere. \`E inoltre possibile modificare le funzioni di trasferimento predefinite o aggiungerne tramite l'apposito file XML, esterno al programma.
\end{itemize}

\subsection{Obiettivi desiderabili}
\begin{itemize}
\item \underline{\textit{D01}} (piani di taglio del volume): soddisfatto. \`E possibile tagliare il volume tramite un \emph{box widget} (quindi composto da 6 piani indipendenti) o con un \emph{plane widget} (quindi un piano singolo, anche rotabile a piacimento);
\item \underline{\textit{D02}} (modifiche alla funzione taglio): soddisfatto. I \emph{widget} per tagliare il volume permettono l'interazione dell'utente per effettuare il taglio desiderato, posizionando e ridimensionando i \emph{widget} a piacimento. \`E inoltre possibile nasconderli o reimpostarli alla loro posizione originale;
\item \underline{\textit{D03}} (ottimizzazione \emph{rendering} GPU): soddisfatto. Il \emph{mapper} GPU con alcune impostazioni particolari si è dimostrato il metodo più performante per effettuare il \emph{render} del volume.
\end{itemize}

\subsection{Obiettivi facoltativi}
\begin{itemize}
	 \item \underline{\textit{F01}} (algoritmi di segmentazione con ITK): non soddisfatto. Non si è dedicato del tempo per studiare approfonditamente ITK, in quanto utilizzare gli algoritmi di segmentazione era solo una possibilità non strettamente richiesta dall'azienda;
	 \item \underline{\textit{F02}} (analisi \emph{unit-testing} su GUI-Qt): soddisfatto. Sono stati sviluppati alcuni semplici \emph{unit-test} utilizzando il \emph{framework} di \emph{test} fornito da Qt. Come accennato nella sezione \nameref{sec:test-begin} (§\ref{sec:test-begin}) non è stato possibile fare un vero e proprio test sul \emph{render} del volume, ma il \emph{tutor} si è trovato d'accordo con questo risultato;
	 \item \underline{\textit{F03}} (\emph{porting} librerie aziendali su CMake): soddisfatto. Le librerie aziendali scelte dal \emph{tutor} sono state portate su CMake, e molte attivamente utilizzate, come la libreria di caricamento di un volume DICOM.
\end{itemize}

%**************************************************************
\section{Conoscenze e abilità acquisite}
Questa esperienza di stage mi ha permesso di apprendere e consolidare diverse nozioni teoriche e pratiche durante i due mesi di stage, che verranno elencate di seguito.

\subsection{Azienda}
Ho avuto modo di imparare come lavora un'azienda che opera in ambito medico: le certificazioni che sono necessarie al \emph{software} per essere distribuito, gli strumenti radiologici utilizzati per effettuare dei test e la gestione delle immagini: dall'archiviazione e la visualizzazione, alla stampa del disco per il paziente. Ho avuto modo di imparare anche come la mia azienda e in particolare il mio \emph{tutor} fanno manutenzione in remoto o di persona presso alcune loro installazioni quando si verifica un problema, e come si rechino direttamente sul luogo per effettuare alcuni test per nuove applicazioni.

\subsection{Immagini radiologiche e DICOM}
Come accennato nella sezione precedente, ho avuto modo di imparare molto di come vengono gestite le immagini radiologiche. Imparare cos'è e come funziona lo standard DICOM: da come riceve le immagini dallo scanner a come vengono archiviate. Leggere nella documentazione tutti gli attributi di cui è composto ed utilizzarlo attivamente è stato molto interessante.

\subsection{Volume Rendering}
Prima di questo stage conoscevo solo delle basi e delle nozioni basilari riguardo il \emph{volume rendering}, studiato per interesse personale. Imparare come viene utilizzato e come si sta espandendo in ambito medico è stato illuminante, ed utilizzarlo personalmente, seppur ad un livello più alto tramite un'astrazione (VTK), è stato estremamente interessante. Studiare ed utilizzare tutto ciò che è necessario per una corretta visualizzazione, dalla correttezza dell'input alle modifiche al volume tramite funzioni di trasferimento, è stata una delle attività più significative. Gli ultimi giorni di stage ho avuto modo anche di analizzare per curiosità personale lo \emph{shader} che VTK utilizza sulla GPU, ed è stato molto entusiasmante poter leggere nel dettaglio come funziona.

\subsection{Linguaggi e tecnologie}
Questo stage mi ha permesso principalmente di migliorare le mie conoscenze di C\texttt{++}, linguaggio già utilizzato sia in Università che a livello personale. Imparare ad utilizzare CMake, già utilizzato in precedenza ma mai scritto personalmente, ha senza dubbio ampliato le mie conoscenze e sono sicuro che mi tornerà molto utile in futuro, in quanto è un sistema di \emph{build} molto diffuso e utilizzato nell'ambito del C\texttt{++}. Migliorare le mie conoscenze di Qt ed utilizzarle per sviluppare un'applicazione completa, seppur concentrata su un \emph{widget} solo, è stato valido, in quanto è un \emph{framework} molto diffuso per sviluppare interfacce grafiche. Infine, imparare ad utilizzare una libreria come VTK, seppur specifica per l'ambito scientifico, è stato molto interessante. 

\subsection{Strumenti}
Migliorare le mie conoscenze e le mie abilità nell'utilizzare strumenti come Visual Studio e Qt Creator è stato senza dubbio utile: imparare meglio come utilizzare il \emph{debugger}, come caricare, compilare e gestire un progetto CMake direttamente dall'IDE sono tutte conoscenze estremamente utili. Lo strumento di controllo versione Git era già stato da me utilizzato per dei progetti universitari, ma imparare come viene utilizzato in un contesto aziendale e con che \emph{software} di utilità è stato senza dubbio arricchente.

%**************************************************************
\section{Valutazione personale}
Personalmente mi ritengo molto soddisfatto dell'esperienza fatta durante lo stage, anche se, lavorando principalmente da casa per colpa dell'attuale emergenza dovuta al Covid, non l'ho vissuto integralmente come hanno fatto altri miei colleghi, che hanno frequentato attivamente l'azienda, con il rapporto e la collaborazione con i colleghi di lavoro ogni giorno. Sono molto fiero dei risultati ottenuti, soprattutto perché quando ho discusso lo stage con il \emph{tutor} la prima volta mi era sembrato un progetto piuttosto complesso da realizzare in due mesi. Con particolare impegno tuttavia sono riuscito a portarlo a termine, e il \emph{tutor} si è dichiarato pienamente soddisfatto del risultato. Gli obiettivi sono stati tutti raggiunti, tranne uno facoltativo; per la prima volta ho sperimentato la soddisfazione di aver concretamente condotto un progetto aziendale dall'inizio alla fine. L'unico obiettivo non completato è quello riguardo ITK, \emph{framework} analizzato a grandi linee e discusso con il \emph{tutor}, ma non realizzato per mancanza di tempo.
\\
Sono anche molto soddisfatto di ciò che ho imparato riguardo l'ambito medico: come funzionano gli esami radiologici tridimensionali, come vengono archiviate e gestite le immagini, come vengono stampati i dischi da dare al paziente, e come sia fondamentale gestire correttamente le immagini che il radiologo andrà ad analizzare. Vedere e provare con mano un monitor radiologico è stata un'esperienza estremamente istruttiva.
\\
Applicare i concetti imparati all'Università (nel mio caso in particolare quelli di
Programmazione ad Oggetti) ad un contesto reale in azienda è stato molto interessante;
da un lato mi sono reso conto di com'è strutturata un'azienda, di come gestisce un progetto, di quali siano le specifiche figure professionali, i ritmi di lavoro, le collaborazioni fra colleghi, le scelte implementative; dall'altro ho verificato tutti gli aspetti necessari da tenere in considerazione per sviluppare una vera applicazione. Ho quindi preso maggiore consapevolezza delle mie conoscenze e di come ampliarle in caso di difficoltà. In sostanza questa esperienza ha rafforzato il mio interesse verso il settore della programmazione grafica, attualmente molto in espansione.

