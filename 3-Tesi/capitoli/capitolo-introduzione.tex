% !TEX encoding = UTF-8
% !TEX TS-program = pdflatex
% !TEX root = ../tesi.tex

%**************************************************************
\chapter{Introduzione}
\label{cap:introduzione}
%**************************************************************

%**************************************************************
\section{L'azienda}

G-Squared Srl è una società di ingegneria informatica specializzata nello sviluppo di software applicativo medicale. Nata nel 2008 con sede legale a Vicenza e come spin-off della società Studio Synthesis Sistemi Avanzati srl, ora ha sede operativa a Ponte San Nicolò (PD).
\\
La filosofia di G-Squared si incentra molto sui dipendenti, concentrandosi su spirito di squadra, rispetto delle regole e valore alla persona e al merito.
\\
Lavorando con software medici, come visualizzatori \textit{\gls{dicom}}, G-Squared è un'azienda certificata ISO 13485:2016. Possiede inoltre molte certificazioni per i propri software, che nel caso delle immagini DICOM si attengono alle relative dichiarazioni di conformità. Più in generale, molti dei software prodotti da G-Squared è stata classificata come dispositivo di classe I secondo i criteri della Direttiva 1993/42/CE.

\begin{figure}[ht]
    \centering
    \includegraphics[width=0.8\textwidth]{immagini/logo-azienda.png}
    \caption{\textit{Logo azienda G-Squared}}
    \textbf{Fonte}: \href{https://www.gsquared.it/it}{gsquared.it}
    \label{fig: Logo azienda G-Squared}
\end{figure}

\newpage
%**************************************************************
\section{Contesto Applicativo}
G-Squared Srl offre vari software e servizi per elaborare, archiviare e distribuire immagini radiologiche. Uno dei più importanti per il mio stage è EyeRad.

\begin{figure}[h]
    \centering
    \includegraphics[width=0.5\textwidth]{immagini/logo-software-eyerad.png}
    \caption{\textit{Logo EyeRad G-Squared}}
    \textbf{Fonte}: \href{https://www.gsquared.it/it/pacs-ris-web-gateway-dicom-ihe-2/software-immagini-medicali/12-eyerad-refertazione-digitale}{gsquared.it/12-eyerad-refertazione-digitale}
    \label{fig: Logo software EyeRad G-Squared}
\end{figure}

EyeRad è un sistema di acquisizione, analisi e gestione di immagini DIGITALI provenienti da sistemi che aderiscono allo standard DICOM in ambiente Microsoft Windows.
EyeRad è in grado di visualizzare, elaborare, archiviare, inviare e stampare immagini digitali.
\\
\'E nato nato per soddisfare le esigenze di efficienza e precisione richieste ad una workstation per la refertazione radiologica: avvalendosi dell’uso di monitor LCD ad alta risoluzione, 2-3-5 MPixel e sfruttando appieno la possibilità di tali monitor di visualizzare le immagini digitali con 2048 toni di grigio.

%**************************************************************
\section{Tecnologie Utilizzate}
\subsection{C++}\label{sec:C++}
C++ è un linguaggio di programmazione general-purpose, sviluppato come evoluzione del linguaggio C inserendo la programmazione orientata agli oggetti, col tempo ha avuto notevoli evoluzioni, come l'introduzione dell'astrazione rispetto al tipo. \'E il linguaggio principale usato da G-Squared, di consequenza la sua scelta è stata praticamente obbligatoria. Si è deciso inoltre di utilizzare lo standard C++17.

\subsection{Qt5}\label{sec:Qt5}
Qt è un \textit{\gls{frameworkg}} open-source multipiattaforma per lo sviluppo di programmi con interfaccia grafica tramite l'uso di \textit{\gls{widgetg}}.

\newpage
%**************************************************************
\section{Strumenti Utilizzati}
\subsection{CMake}\label{sec:cmake}
CMake è un software open-source multipiattaforma per l'automazione dello sviluppo il cui nome è un'abbreviazione di cross platform make. Questo software nasce per rimpiazzare Automake nella generazione dei Makefile, cercando di essere più semplice da usare. Infatti, nella maggior parte dei progetti, non esiste un Makefile incluso nei sorgenti, dato che questo non è portabile.

\subsection{Visual Studio}\label{sec:visual-studio}
Microsoft Visual Studio, chiamato più comunemente Visual Studio, è un ambiente di sviluppo integrato (Integrated development environment o IDE) sviluppato da Microsoft.
\\
Visual Studio è multilinguaggio, ma è stato utilizzato con C++.

\subsection{Qt Creator}\label{sec:qt-creator}
Qt Creator è un è un IDE open-source, che semplifica lo svoliluppo di applicazioni con una GUI. Fa parte dell'SDK di Qt ed è ben integrato con tutti i relativi strumenti. Permette la scelta del compilatore, che nel mio caso è stato MSVC, in modo da utilizzare lo stesso compilatore tra Visual Studio e Qt Creator.
\\
\'E il principale IDE utilizzato da G-Squared, quindi era importante verificare che tutto funzionasse anche su Qt Creator.

%**************************************************************
\section{Organizzazione del testo}

Riguardo la stesura del testo, relativamente al documento sono state adottate le seguenti convenzioni tipografiche:
\begin{itemize}
	\item gli acronimi, le abbreviazioni e i termini ambigui o di uso non comune menzionati vengono definiti nel glossario, situato alla fine del presente documento;
	\item per la prima occorrenza dei termini riportati nel glossario viene utilizzata la seguente nomenclatura: \emph{parola}\glsfirstoccur;
	\item i termini in lingua straniera o facenti parti del gergo tecnico sono evidenziati con il carattere \emph{corsivo}.
\end{itemize}