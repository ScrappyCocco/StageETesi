% !TEX encoding = UTF-8
% !TEX TS-program = pdflatex
% !TEX root = ../tesi.tex

%**************************************************************
\chapter{Volume Rendering e Qt}
\label{cap:teoria-stage}
%**************************************************************

%**************************************************************
\section{Volume Rendering}
\intro{Concetti di base, teoria, algoritmi e gestione/caricamento immagini DICOM}\\

%**************************************************************
\section{VTK}
\intro{Funzionalità libreria, rendering, taglio, CPU/GPU e funzioni di trasferimento}\\

%**************************************************************
\section{Integrazione con Qt}

%**************************************************************
\section{Software}
\intro{Probabile menzione a Slicer3D, il principale software preso come riferimento}\\

%**************************************************************
\section{Strumenti CTK}

%**************************************************************
\section{Basi di ITK}

