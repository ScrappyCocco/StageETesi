% !TEX encoding = UTF-8
% !TEX TS-program = pdflatex
% !TEX root = ../tesi.tex

%**************************************************************
\chapter{Descrizione dello stage}
\label{cap:descrizione-stage}
%**************************************************************

%**************************************************************
\section{Scelta dello stage}
Da anni ho iniziato ad interessarmi di più al rendering real-time e alle sue varie applicazioni, dal gaming a simulazioni di vario genere.
Anche avendo partecipato all'evento Stage-IT 2019, durante la ricerca di uno stage a fine 2019 ho deciso di indagare più in profondità, cercando le opportunità presentate agli eventi Stage-IT degli anni passati. \\
Dopo varie ricerche, ho trovovato G-Squared nel documento di Stage-IT 2017. Contattando e confrontandomi con l'azienda, ho deciso di unire la mia passione per C\texttt{++} e per il rendering a un ambito utile, come quello medicale, in modo da imparare di più su come funziona e tutte le tecnologie che vengono utilizzate.

%**************************************************************
\section{Introduzione al progetto}

%**************************************************************
\section{Obiettivi}
\subsection{Obiettivi obbligatori}\label{sec:obiettivi-obbligatori}

\subsection{Obiettivi desiderabili}\label{sec:obiettivi-desiderabili}

%**************************************************************
\section{Pianificazione e metodologia}

%**************************************************************
\section{Aspettative aziendali}


%**************************************************************
\section{Aspettative personali}