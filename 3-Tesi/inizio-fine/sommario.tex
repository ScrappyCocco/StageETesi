% !TEX encoding = UTF-8
% !TEX TS-program = pdflatex
% !TEX root = ../tesi.tex

%**************************************************************
% Sommario
%**************************************************************
\cleardoublepage
\phantomsection
\pdfbookmark{Sommario}{Sommario}
\begingroup
\let\clearpage\relax
\let\cleardoublepage\relax
\let\cleardoublepage\relax

\chapter*{Sommario}

Il presente documento descrive il lavoro svolto durante il periodo di stage, della durata di 304 ore, dal laureando Michele Roverato presso l'azienda G-Squared Srl di Ponte San Nicolò (PD).\\
Il progetto di stage prevedeva l'analisi, la progettazione e lo sviluppo di un Widget Qt che permetta la visualizzazione 3D volumetrica di una ricostruzione fatta tramite immagini diagnostiche medicali radiologiche (CT, MR, MG).\\
Il documento è così suddiviso:
\begin{itemize}
    \item \hyperref[cap:introduzione]{Il primo capitolo} descrive l'azienda presso cui ho svolto lo stage. In particolare viene illustrata la sua storia e i suoi prodotti;
    \item \hyperref[cap:descrizione-stage]{Il secondo capitolo} descrive gli obiettivi dello stage in relazione alle aspettative aziendali e personali;
    \item \hyperref[cap:teoria-stage]{Il terzo capitolo} descrive la teoria, i software e gli strumenti studiati ed utilizzati per svolgere lo stage;
    \item \hyperref[cap:resoconto-stage]{Il quarto capitolo} descrive nel modo più completo possibile l’esperienza effettuata nella progettazione e nello sviluppo del progetto di stage;
	\item \hyperref[cap:conclusioni]{Il quinto capitolo} presenta una valutazione dello stage in relazione agli obiettivi dell'azienda e all'esperienza da me acquisita nel corso del suo svolgimento.
\end{itemize}

\endgroup			

\vfill

